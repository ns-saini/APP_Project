\documentclass{report}
\usepackage{algorithmicx, algpseudocode, graphicx, imakeidx, hyperref}
\usepackage[a4paper, total={6in, 8in}]{geometry}

\makeindex
\begin{document}
\title{CHEERS}
\author{Team E}
\date{22 Feb 2023}
\maketitle
\tableofcontents{}
\printindex{}

\chapter{Outline}
\section{Introduction}
  Coasters prevent condensation from dripping along the glass, which can damage the surface of tables. The project referred to as \textbf {CHEERS} deals with two circular coasters overlapping each other. The objective of \textbf {CHEERS} is to compute the diameter of the overlapping region such that 
  the real estate is half that of any of the coasters.

\section{Background}
  In this project, we determine the roots of an equation to calculate the angle made at the vertex of a coaster using the \textbf{Secant approximation method}. This method approximates the value of a function using a secant line passing through two points on a graph.
  $$ x_{n+1} = x_n - \frac{f(x_n) \cdot (x_n - x_{n-1})}{f(x_n) - f(x_{n-1})} $$

  \vspace*{20pt}
  \noindent The Sin/Cos value is approximated using the \textbf{Mclaurin Series} which is the sum of derivatives of a function.
  $$\sin(x) = \sum_{n=0}^{\infty} \frac{(-1)^n}{(2n+1)!}x^{2n+1}$$
  $$\cos(x) = \sum_{n=0}^{\infty} \frac{(-1)^n}{(2n)!}x^{2n}$$ 

\section{Scope}
  The length \textbf{l} of the overlapping coaster region can be computed by the equation
  $$l = 2R\left(1 - \cos\frac{\alpha}{2}\right)$$

  \indent R \textrightarrow \;radius of the coaster \\
  \indent $\alpha$ \textrightarrow \;angle created with the vertex at the centre point of the left coaster

  \vspace{20pt}
  The angle, $\alpha$ can be calculated using
  $$\alpha - \sin(\alpha) = \frac{\pi}{2}$$
  
  \vspace*{20pt}
  
  \begin{itemize}
    \item {The roots of the equation need to be determined to compute the value of $\alpha$. We use the Secant approximation method to arrive at a solution.}
    \item {The Mclaurin series has been used to approximate the Sin and Cos values and, subsequently, calculate l.}
  \end{itemize}

\section{Objectives}
\begin{itemize}
  \item To compute the value of $\Pi$.
  \item To compute the exponent of a number using recursion.
  \item To compute the factorial of a number using recursion.
  \item To approximate Sin and Cos using Mclaurin Series.
  \item To find the roots of an equation using Secant Approximation.
\end{itemize}

\section{Assumptions}
  The initial guess values for approximating the roots of an equation will be pre-determined.

\section{Roles \& Responsibilities}
  \begin{center}
    \begin{tabular}{ |p{4cm}|p{4cm}|p{4cm}|  }
      \hline
      \multicolumn{3}{|c|}{\textbf{TEAM E}} \\
      \hline
      \textbf{Student ID}& \textbf{Name} & \textbf{Responsibilities} \\
      \hline
      40195801 & Nishant Saini & Algorithms \\
      \hline
      40207625 & Pouria Pirian & XML \\
      \hline
      40226505 & Omer Saeym & Input Validation \\
      \hline
      40218678 & Sachin Prakash & CRC cards, Pseudocode \\
      \hline
      40234194 & Chandra Sagar Reddy & Evidence, Referencing \\
      \hline
      40230025 & Kajal Sehrawat & Controller   \\
      \hline
      \end{tabular}
  \end{center}

\chapter{CRC Cards}
\begin{figure}[h!]
    \centering
    \begin{tabular}{@{}c@{}}
      \includegraphics[width=.3\linewidth]{resources/Trigonometry.png} 
        \hspace*{30pt}
      \includegraphics[width=.3\linewidth]{resources/MathLib.png}
    \end{tabular}
  
    \begin{tabular}{@{}c@{}}
        \includegraphics[width=.3\linewidth]{resources/RootApproximation.png} 
          \hspace*{30pt}
        \includegraphics[width=.3\linewidth]{resources/Validator.png}
      \end{tabular}

      \begin{tabular}{@{}c@{}}
        \includegraphics[width=.3\linewidth]{resources/ErrorHandler.png} 
          \hspace*{30pt}
        \includegraphics[width=.3\linewidth]{resources/OutputGenerator.png}
      \end{tabular}
      
      \begin{tabular}{@{}c@{}}
        \includegraphics[width=.4\linewidth,height=90pt]{resources/Controller.png}
      \end{tabular}

      \vspace{\floatsep}
    \caption{CRC Cards for CHEERS}\label{fig:myfig}
\end{figure}

\chapter{Pseudocode}
\section{Sin using Mclaurin}
\begin{flushleft}
We arrive at a value for sin of a number using the Mclaurin series expansion.
\end{flushleft}
% \break
\begin{algorithmic}[1]
    \Function{calculate\_sin}{$rad$}
        \State $total\_iterations \gets precision$
        \State $sin\_val \gets 0$
        \For{$i \gets 0, total\_iterations$}
            \State $term \gets (-1)^i \cdot rad^{2i + 1} / (2i + 1)!$
            \State $sin\_val \gets sin\_val + term$
        \EndFor
        \State \Return $sin\_val$
    \EndFunction
\end{algorithmic}

\section{Secant Approximation}
\begin{flushleft}
The roots of an equation can be derived using the secant approximation method.
\end{flushleft}
% \break
\begin{algorithmic}[1]
\Function{secant\_approximation}{func, $x_0$, $x_1$, $e$}
    \State $x_2 \gets 0$
    \State $step \gets 1$
    \While{True}
        \If{func($x_0$) == func($x_1$)}
            \State \textbf{break}
        \EndIf
        \State $x_2 \gets x_0 - (x_1 - x_0) * func(x_0) / (func(x_1) - func(x_0))$
        \State $x_0 \gets x_1$
        \State $x_1 \gets x_2$
        \State $step \gets step + 1$
        \If{$step > num\_terms$}
            \State \textbf{print}("Not convergent")
            \State \textbf{break}
        \EndIf
        \If{func($x_2$) - func($x_1$) $>$ e}
            \State \textbf{break}
        \EndIf
    \EndWhile
    \State \textbf{return} $x_2$
\EndFunction
\end{algorithmic}

\section{Pi Calculation}
\begin{flushleft}
  The algorithm to calculate the value of Pi is the following.
\end{flushleft}
% \break
\begin{algorithmic}[1]
    \Function{CalculatePi}{}
        \State $val \gets 0.0$
        \State $total\_terms \gets 100$
        \For{$i=1$ to $2*total\_terms$}
            \State $sign \gets -(i\%4-2)$
            \State $val \gets val + \frac{sign}{i}$
        \EndFor
        \State \Return $4 * val$
    \EndFunction
\end{algorithmic}

\section{Exponent}
\begin{flushleft}
The exponent of a number can be obtained using the following algorithm.
\end{flushleft}
% \break
\begin{algorithmic}[1]
\Function{exp}{$number$, $power$}
    \If{$power = 0$}
        \State \Return $1$
    \EndIf
    \State $temp \gets$ \Call{exp}{$number$, $\lfloor power/2 \rfloor$}
    \If{$power$ is even}
        \State \Return $temp * temp$
    \Else
        \State \Return $temp * temp * number$
    \EndIf
\EndFunction
\end{algorithmic}

\section{Factorial}
\begin{flushleft}
This is the algorithm used to obtain the factorial of a number.
\end{flushleft}
% \break
\begin{algorithmic}[1]
\Function{factorial}{$num$}
  \If{$num < 0$}
    \State \textbf{raise} Exception(``Factorial can't be calculated for negative numbers'')
  \EndIf
  \State $result \gets 1$
  \For{$i \gets 1$ \textbf{to} $num$}
    \State $result \gets result \times i$
  \EndFor
  \State \textbf{return} $result$
\EndFunction
\end{algorithmic}

\printindex
\end{document}