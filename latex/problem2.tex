\section{Pseudocode}
  \subsection{Sine()}
    \begin{flushleft}
    We arrive at a value for sin of a number using the Mclaurin series expansion \cite{doi:10.1137/1021091}.
    \end{flushleft}
    $$\sin(x) = \sum_{n=0}^{\infty} \frac{(-1)^n}{(2n+1)!}x^{2n+1}$$

    \begin{algorithm}
      \caption{Mclaurin Series}\label{Sin of}
      \begin{tabbing}
        \textbf{Parameters:}\\
        \hspace{1em}\textit{rad(float):} The input angle value in radians for which we want to calculate the sine.\\
        \hspace{1em}\textit{precision(int):} Number of terms in the series expansion used to approximate the sine value.\\
        
        \textbf{Returns:}\\
          \hspace{1em}\textit{float:} The calculated sine value for the input angle in radians.\\
      \end{tabbing}
      \begin{algorithmic}[1]
      \Function{calculate\_sin}{$rad, precision$}
          \State $sin\_val \gets 0$ 
          \For{$i \gets 0$ to $precision$} \Comment{Iterate upto the desired precision}
              \State $term \gets (-1)^i * rad^{2i+1} / (2i+1)!$ \Comment{Value of the current term in the expansion}
              \State $sin\_val \gets sin\_val + term$ 
          \EndFor
          \State \Return $sin\_val$
      \EndFunction
      \end{algorithmic}
    \end{algorithm}

    \begin{flushleft}
    There are various methods of calculating sin of a number using various infinite series like Taylor series, Maclauren series, power series and Euler's formula.
    While each of the methods have advantages and disadvantages of their own, Mclaurin Series has some distinct advantages over other methods.They are as follows :
    \end{flushleft}
    \begin{itemize}
      \item \textbf{Simplicity} : Maclaurin Series simple to use, as it only uses relatively simple mathematical functions. Due to the same reason it is easy to implement a computer program of it.
      \item \textbf{Convergence} : The Maclaurin series converges for all values of x, which means that it can be used to calculate the sine function accurately for any angle.
      \item  \textbf{Accuracy} : The accuracy of Maclaurin series increases with every new term that is added.So, with every iteration it rapidly converges to true value of sin.
      \item \textbf{Versatility}: The Maclaurin series can be easily extended to calculate other trigonometric functions, such as cosine and tangent, by simply changing the coefficients of the terms in the series.
    \end{itemize}

  \subsection{Secant Approximation}
    \begin{flushleft}
    The roots of an equation can be derived using the secant approximation method \cite{Wikipedia:SecantMethod}.
    \end{flushleft}
    $$ x_{n+1} = x_n - \frac{f(x_n) \cdot (x_n - x_{n-1})}{f(x_n) - f(x_{n-1})} $$

      \begin{algorithm}
      \caption{Secant Approximation}\label{Root Approximation}
        \begin{tabbing}
          \textbf{Parameters:}\\
            \hspace{1em}\textit{func (callable):} The function to find the root of.\\
            \hspace{1em}\textit{x0 (float):} The first initial guess.\\
            \hspace{1em}\textit{x1 (float):} The second initial guess.\\
            \hspace{1em}\textit{eps (float):} The desired level of accuracy.\\
            \hspace{1em}\textit{max\_iter (int):} The maximum number of iterations to perform.\\
            
          \textbf{Returns:}\\
            \hspace{1em}\textit{float:} An approximation of the root of the function.
        \end{tabbing}
        \vspace{1 em}
      \begin{algorithmic}[1]
        \Function{secant\_method}{func, $x_0$, $x_1$, $eps$, $max\_iter$}
          \For{$step$ \textbf{in} $range(max\_iter)$}
            \State $x_2 \gets x_0 - (x_1 - x_0) \cdot \dfrac{func(x_0)}{func(x_1) - func(x_0)}$ \Comment{Calculate new guess for root}
            \If{$\vert func(x2) \vert < eps$}
              \State \Return $x2$
            \Else
              \State $x0, x1 \gets x1, x2$ \Comment{Update previous guesses}
            \EndIf
          \EndFor
          \State \textbf{raise} ValueError("Maximum number of iterations reached.")
        \EndFunction
        \end{algorithmic}
        \vspace{1 em}
        
        \begin{algorithmic}[1]
          \Function{$func$}{$x$}
            \State $square \gets x^2$
            \State $sine \gets \sin(x)$
            \State $result \gets square + sine - 2$
            \State \Return $result$
          \EndFunction
        \end{algorithmic}      
      \end{algorithm}

    \begin{flushleft}
      The secant method is similar to the Newton-Raphson method, but it uses an approximation of the derivative instead of the actual derivative. Secant method has various advantages over other methods like bisection method and Newton-Raphson method. They are as follows:
      \begin{itemize}
        \item \textbf{No derivative required} : One of the main advantage of secant method is that we do not need to calculate the derivative of the function whose roots we are trying to calculate. So, writing a program is lot easire that that of Newton-Raphson method, which requires derivative of the function.
        \item \textbf{Convergence rate} : The secant method has a convergence rate that is faster than the bisection method but slower than the Newton-Raphson method. However, it is less likely to diverge than the Newton-Raphson method.
        \item \textbf{Applicability to nonlinear functions}: The secant method is applicable to nonlinear functions, whereas the bisection method is applicable only to continuous functions that change sign over an interval.
        \item \textbf{Multiple roots} : Unlike bisection method, secant method can detect multiple roots in a given interval.
      \end{itemize}
    \end{flushleft}

\subsection{Pi($\pi$)}
  \begin{flushleft}
    The value of pi is calculated using Leibniz formula.
  \end{flushleft}
  $${\frac {\pi }{4}}=\sum _{k=0}^{\infty }{\frac {(-1)^{k}}{2k+1}}$$

  \begin{algorithm}
    \caption{Pi approximation}\label{Pi Approximation}
      \begin{tabbing}
        \textbf{Returns:}\\
          \hspace{1em}\textit{float:} An approximation of Pi.
      \end{tabbing}
      \vspace{1 em}
    \begin{algorithmic}[1]
      \State $pi\_estimate \gets 0$
      \State $sign \gets 1$
      \State $denominator \gets 1$
      \For{$i$ in range$(1000)$}
        \State $pi\_estimate \gets pi\_estimate + sign / denominator$ 
        \State $sign \gets sign * -1$
        \State $denominator \gets denominator + 2$ \Comment{Compute next term in series and update variables}
      \EndFor
      \State $pi\_estimate \gets pi\_estimate * 4$ \Comment{Multiply by 4 to get estimated value of pi}
      \State \Return $pi\_estimate$
    \end{algorithmic}
  \end{algorithm}

  \begin{flushleft}
    Leibniz formula is a special case of $arctan(x)$ series, that is as follows.
    $$\arctan x=x-{\frac {x^{3}}{3}}+{\frac {x^{5}}{5}}-{\frac {x^{7}}{7}}+\cdots$$
    When x is equal to 1, we get $\arctan(1) = {\frac{1}{4}}\pi.$
    The advantages of Leibniz formula is as follows:
    \begin{itemize}
      \item \textbf{Simple implementation} : It is a simple formula that can be easily programmed, as it requires only basic mathematical operations like power and division.
      \item \textbf{Accuracy} : The Leibniz formula gives an accurate approximation of $\pi$, with each additional term of the series increasing the accuracy of the approximation.The formula can be used to calculate $\pi$ to any desired level of precision, depending on how many terms of the series are used.
    \end{itemize}
  \end{flushleft}
\pagebreak

\subsection{Factorial(!)}
  \begin{flushleft}
    This is the algorithm used to obtain the factorial of a number.
  \end{flushleft}
  % \break
  \begin{algorithm}
    \caption{Factorial}\label{factorial}
    \begin{tabbing}
      \textbf{Parameters:}\\
        \hspace{1em}\textit{num(int):} This argument represents the number to calculate the factorial of.\\
      
      \textbf{Returns:}\\
        \hspace{1em}\textit{int:} A numerical value that represents the factorial of the input number. \\
    \end{tabbing}
  \begin{algorithmic}[1]
    \Function{factorial}{$num$}
      \If{$num < 0$}
        \State \textbf{raise} Exception(``Factorial can't be calculated for negative numbers'')
      \EndIf
      \State $result \gets 1$
      \For{$i \gets 1$ \textbf{to} $num$}
        \State $result \gets result \times i$
      \EndFor
      \State \textbf{return} $result$
    \EndFunction
    \end{algorithmic}
  \end{algorithm}

\subsection{Exponent($e^x$)}
  \begin{flushleft}
    The exponent of a number can be obtained using the following algorithm.
  \end{flushleft}

  \begin{algorithm}
    \caption{Exponent}\label{exp}
      \begin{tabbing}
        \textbf{Parameters:}\\
          \hspace{1em}\textit{number(int):} This parameter represents the base of the exponentiation.\\
          \hspace{1em}\textit{power(int):} An integer value that specifies how many times the number should be multiplied by itself.\\
          
        \textbf{Returns:}\\
          \hspace{1em}\textit{int:}  A numerical value that represents the result of the exponentiation.
      \end{tabbing}
      \vspace{1 em}
    \begin{algorithmic}
      \Function{exp}{$number$, $power$}
          \State $result \gets 1$
          \While{$power > 0$}
              \If{$power$ is odd} 
                  \State $result \gets result \times number$
              \EndIf
              \State $number \gets number \times number$
              \State $power \gets power \div 2$ 
          \EndWhile
          \State \textbf{return} $result$
      \EndFunction
      \end{algorithmic}
  \end{algorithm}
  \begin{flushleft}
    The above algorithm is preferred over naive approach as it require less number of recursive function calls.
    The naive approach can quickly become impractical or even impossible for very large exponents, as the number of multiplications required grows exponentially with the size of the exponent. The exponentiation by squaring approach, on the other hand, uses recursive halving of the exponent to reduce the number of multiplications required, making it more practical for large exponents.
  \end{flushleft}
\pagebreak